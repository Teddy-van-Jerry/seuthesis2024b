\ExplSyntaxOff

\usetikzlibrary{positioning, backgrounds}

\pgfmathsetmacro{\jumpswap}{1}
\tikzset{
  % set up keys for radius, position, swap
  jump radius/.estore in=\jumpradius,
  jump pos/.estore in=\jumppos,
  jump swap/.code={\pgfmathsetmacro{\jumpswap}{\jumpswap*-1}},
  jump radius=0.8mm,
  jump pos=0.5,
  % set up styles for the various to-paths
  -u-/.style={% straight line
    to path={
      let \p1=(\tikztostart),\p2=(\tikztotarget),\n1={atan2(\y2-\y1,\x2-\x1)} in
      (\p1) -- ($($(\p1)!\jumppos!(\p2)$)!\jumpradius!(\p1)$)
      arc[start angle=\n1+180,delta angle=-180*\jumpswap,radius=\jumpradius] -- (\p2)}
  },
  -u|/.style={% -| path with jump on horizontal leg
    to path={
      let \p1=(\tikztostart),\p2=(\tikztostart-|\tikztotarget), \n1={atan2(\y2-\y1,\x2-\x1)} in
      (\p1) -- ($($(\p1)!\jumppos!(\p2)$)!\jumpradius!(\p1)$)
      arc[start angle=\n1+180,delta angle=-180*\jumpswap,radius=\jumpradius] --(\p2) -- (\tikztotarget)}
  },
  |u-/.style={% |- path with jump on vertical leg
    to path={
      let \p1=(\tikztostart),\p2=(\tikztostart|-\tikztotarget), \n1={atan2(\y2-\y1,\x2-\x1)} in
      (\p1) -- ($($(\p1)!\jumppos!(\p2)$)!\jumpradius!(\p1)$)
      arc[start angle=\n1+180,delta angle=-180*\jumpswap,radius=\jumpradius] -- (\p2) -- (\tikztotarget)}
  },
  -|u/.style={% -| path with jump on vertical leg
    to path={
      let \p1=(\tikztostart-|\tikztotarget),\p2=(\tikztotarget), \n1={atan2(\y2-\y1,\x2-\x1)} in
      (\tikztostart) -- (\p1) -- ($($(\p1)!\jumppos!(\p2)$)!\jumpradius!(\p1)$)
      arc[start angle=\n1+180,delta angle=-180*\jumpswap,radius=\jumpradius] -- (\p2)}
  },
  |-u/.style={% |- path with jump on horizontal leg
    to path={
      let \p1=(\tikztostart|-\tikztotarget),\p2=(\tikztotarget), \n1={atan2(\y2-\y1,\x2-\x1)} in
      (\tikztostart) -- (\p1) -- ($($(\p1)!\jumppos!(\p2)$)!\jumpradius!(\p1)$)
      arc[start angle=\n1+180,delta angle=-180*\jumpswap,radius=\jumpradius] -- (\p2)}
  },
  % define the jump style, set it to use straight line by default
  jump/.style={-u-,#1},
  jump/.default={}
}

% similar to env "pgfonlayer", but the latest contents are typeset on
% lowest bottom (on reversed order)

\let\pgfonlayerreversed\pgfonlayer
\let\endpgfonlayerreversed\endpgfonlayer

\RequirePackage{xpatch}
\xpatchcmd\pgfonlayerreversed
  {\expandafter\box\csname pgf@layerbox@#1\endcsname\begingroup}
  {\begingroup}
  {}{\fail}

\xpatchcmd\endpgfonlayerreversed
  {\endgroup}
  {\endgroup\expandafter\box\csname pgf@layerbox@\pgfonlayer@name\endcsname}
  {}{\fail}

\tikzset{
  on background layer reversed/.style={%
    execute at begin scope={%
      \pgfonlayerreversed{background}%
      \let\tikz@options=\pgfutil@empty
      \tikzset{every on background layer/.try,#1}%
      \tikz@options
    },
    execute at end scope={\endpgfonlayerreversed}
  }
}

\def\StartDrawOnBottomOfLayerStack{%
  \scope\relax
  % patch \path variants to auto insert "\scoped[on lowest layer]"
  % currently \node, \pic, \coordinate, and \matrix are patched
  \let\tikz@path@overlay\tikz@path@overlay@autoscoped
  \let\tikz@path@overlayed\tikz@path@overlayed@autoscoped
}

\def\EndDrawOnTopOfLayerStack{%
  \endscope
}

\def\tikz@path@overlay@autoscoped#1{%
  \let\tikz@signal@path=\tikz@signal@path% for detection at begin of matrix cell
  \pgfutil@ifnextchar<%
    {\tikz@path@overlayed{#1}}
    {\scoped[on background layer reversed] \path #1}}%
\def\tikz@path@overlayed@autoscoped#1<#2>{%
  \scoped[on background layer reversed] \path<#2> #1}%

\ExplSyntaxOn
