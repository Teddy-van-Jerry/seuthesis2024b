\documentclass[fontset = mac ms]{seuthesis2024b}

\title{东南大学本科毕设 \LaTeX{} 模版}
% \title{东南大学本科毕设 \LaTeX{} 模版}[第二行内容]
\studentID{12345678}
\author{泰迪熊}
\department{泰迪熊学院}
\major{毛茸茸工程}
\supervisor{Ted Roosevelt}
\date{2023年12月 至 2024年5月}

\addbibresource{ref/IEEEfull.bib}
\addbibresource{ref/ref.bib}

\CNabstract{%
  此 \LaTeX{} 模板基于东南大学教务处 2024 年 1 月最新 MS Word 版本制作。

  摘要内容独立于正文而存在,是论文内容高度概括的简要陈述,
  应准确、具体、完整地概括论文的主要信息,内容包括研究目的、方法、过程、成果、结论及主要创新之处等,
  不含图表,不加注释,具有独立性和完整性,一般为400字左右。

  “摘要”用三号黑体加粗居中,“摘”与“要”之间空4个半角空格。摘要正文内容用小四号宋体,固定1.5倍行距。

  论文的关键词是反映毕业设计(论文)主题内容的名词,一般为3-5个,排在摘要正文部分下方。关键词与摘要之间空一行。
  关键词之间用逗号分开,最后一个关键词后不加标点符号。
} {关键词1,关键词2,关键词3,关键词4}

\ENabstract{%
  English abstract should correspond to the contents in the Chinese abstract.
  It should describe the thesis contents as a third-person perspective.
  The basic tense used in the abstract is the present tense.
  Other tenses like the past tense or the completion tense should only be used when necessary.

  ABSTRACT should be centered and bolded with Times New Roman \texttt{\string\zihao\{3\}}.

  The English abstract contents format is the same as the Chinese abstract.
  The English ``KEY WORDS'' should be in all uppercase,
  and all keywords have the first letter in capical.
  Keywords are separated by an English comma.
} {Keywords 1, Keywords 2, Keywords 3, Keywords 4}

\begin{document}
  \maketitle

  \chapter{绪论}
    \section{课题背景和意义}
      绪论部分主要论述选题的意义、国内外研究现状以及本文主要研究的内容、研究思路以及内容安排等。

      章标题为三号黑体加粗居中;一级节标题(如,2.1 本文研究内容):四号黑体居左;二级节标题(如,2.1.1 实验方法):小四号宋体居左。

      正文部分为小四号宋体,行间距1.5倍行距,首行缩进2个字符。正文一般不少于15000字。

    \section{研究现状}
      ……

    \section{本文研究内容}
      ……

    \chapter{正文}

      具体研究内容每一章应另起页书写书写,层次要清楚,内容要有逻辑性,每一章标题需要按论文实际研究内容进行填写,不可直接写成第二章 正文。研究内容因学科、选题特点可有差异,但必须言之成理,论据可靠,严格遵循本学科国际通行的学术规范。

      中文为小四号宋体,英文及数字为小四号Times New Roman,首行缩进2个字符,行间距为1.5倍。

    \section{插图格式要求}

      插图力求精炼,且每个插图均应有图序和图名。图序与图名位于插图下方,图序一般按章节编排,如图1-1(第一章第1个图),在插图较少时可以全文连续编序,如图10。
      
      如一个插图由两个及以上的分图组成,分图用(a)、(b)、(c)等标出,并标出分图名。

      简单文字图可用 \LaTeX{} 自带宏包 Ti\textit{k}Z 直接绘制,
      复杂的图考虑使用 standalone 的 Ti\textit{k}Z 完成,以提高图形表达质量。

      插图居中排列,与上文文本之间空一行。图序图名设置为五号宋体居中,图序与图名之间空一格。
      可爱的图~\ref{fig:duck}。
      \begin{figure}[htbp]
        \includegraphics{example-image-duck}
        \caption{经典 \LaTeX{} 鸭子}
        \label{fig:duck}
      \end{figure}

      图~\ref{fig:subfigs} 包含两个子图:
      图~\subref*{subfig:s1} 和 \subref*{subfig:s2}。
      \begin{figure}[htbp]
        \subfloat[子图一\label{subfig:s1}]{\includegraphics[width=.3\linewidth]{example-image-a}}\quad
        \subfloat[子图二\label{subfig:s2}]{\includegraphics[width=.3\linewidth]{example-image-b}}
        \caption{两个子图}
        \label{fig:subfigs}
      \end{figure}

    \section{表格格式要求}

      表格的结构应简洁,一律采用三线表,应有表序和表名,且表序和表名位于表格上方。
      表格可以逐章单独编序(如:表2.1),也可以统一编序(如:表10),采用哪种方式应和插图及公式的编序方式统一。表序必须连续,不得重复或跳跃。

      表格无法在同一页排版时,可以用续表的形式另页书写,续表需在表格右上角表序前加“续”字,如“续表2.1”,并重复表头。

      表格居中,边框为黑色直线1磅,中文为五号宋体,英文及数字为五号Times New Roman字体,表序与表名之间空一格,表格与下文之间空一行。
    
      表格例子如表~\ref{tab:font-effect} 所示。
      表格内字号默认为五号(10.5pt),即 \texttt{\string\small} 或 \texttt{\string\zihao\{5\}}。

      \begin{table}[htbp]
        \caption{字体族、字体形状和字体系列的组合效果}
        \label{tab:font-effect}
        \begin{tabular}{ccccc}
          \toprule
          & & \verb|\itshape| & \verb|\bfseries| & \verb|\itshape\bfseries| \\
          \midrule
          \verb|\rmfamily| & \rmfamily 罗马体 roman & \rmfamily\itshape 倾斜 italic & \rmfamily\bfseries 加粗 bold & \rmfamily\itshape\bfseries 粗斜 bold-italic \\
          \verb|\sffamily| & \sffamily 无衬线 sans  & \sffamily\itshape 倾斜 italic & \sffamily\bfseries 加粗 bold & \sffamily\itshape\bfseries 粗斜 bold-italic \\
          \verb|\ttfamily| & \ttfamily 打字机 mono  & \ttfamily\itshape 倾斜 italic & \ttfamily\bfseries 加粗 bold & \ttfamily\itshape\bfseries 粗斜 bold-italic \\
          \bottomrule
        \end{tabular}
      \end{table}

    \section{表达式}

      论文中的公式应注序号并加圆括号,序号一律用阿拉伯数字连续编序(如 (28) )或逐章编序(如 (3.6) ),
      编号方式应与插图、表格方式一致。序号排在版面右侧,且距右边距离相等。公式与序号之间不加虚线。

      长公式在一行无法写完的情况下,原则上应在等号(或数学符号,如“$+$”、“$-$”号)处换行,数学符号在换行的行首。

      公式及文字中的一般变量(或一般函数)(如坐标$X$、$Y$,电压$V$,频率$f$)宜用斜体,矢量用粗斜体如$\bm{S}$或白斜体上加单箭头,
      常用函数(如三角函数$\cos$、对数函数$\log$等)、数字运算符、化学元素符号及分子式、单位符号、产品代号、人名地名的外文字母等用正体。

      公式排版直接使用 \texttt{equation} 环境,如公式~\eqref{eq:einstein}。
      \begin{equation}\label{eq:einstein}
        E=mc^2.
      \end{equation}

    \section{注释}

      正文中有个别名词或情况需要解释时,可加注说明,注释采用页末注(将注文放在加注页的下端)。
      在引文的右上角标注序号①、②、……,如“马尔可夫链\footnote{马尔可夫链表示……}”。若在同一页中有两个以上的注时,按各注出现的先后,顺序编号。
      引文序号,以页为单位,且注释只限于写在注释符号出现的同页,不得隔页。
    
      注释采用小五号宋体,英文及数字为小五号Times New Roman字体,利用 \texttt{\string\footnote} 插入。

    \section{参考文献}

      算法方面的研究包括:
      OMPL-SBL 算法\cite{zhao2023ompl},
      RIS 的波束设计方法\cite{you2023beam};
      硬件设计方面的研究包括:
      高层级综合工具 FLAMES 库设计\cite{zhao2023flexible},
      自动生成语言 AHDW\cite{zhao2023automatic}。

      泰迪熊的所有工作有\cite{zhao2023ompl,you2023beam,zhao2023flexible,zhao2023automatic}。

    \section{章节}

      在 \LaTeX{} 中,
      章(一级标题)使用 \texttt{\string\chapter},
      节(二级标题)使用 \texttt{\string\section},
      小节(三级标题)使用 \texttt{\string\subsection},
      子小节(四级标题)使用 \texttt{\string\subsubsection},
      段落使用 \texttt{\string\paragraph}。
      官方 MS Word 模板仅提供第一、二级格式规范。
      二级标题及以上会出现在目录中,
      三级标题及以上会出现在 PDF 的 bookmarks 中。

      \subsection{三级标题}
        三级标题内容
        \subsubsection{四级标题}
        四级标题内容
          \paragraph{段落。}内容。
    
  \chapterBib

  \appendix
  \chapter{附录名称}
    对于一些不宜放入正文中、但作为毕业设计(论文)又不可残缺的组成部分或具有重要参考价值的内容,
    可编入毕业设计(论文)的附录中,例如,正文内过于冗长的公式推导、
    方便他人阅读所需的辅助性数学工具或表格、重复性数据和图表、非常必要的程序说明和程序全文、关键调查问卷或方案等。

    附录的格式与正文相同,如有多个附录需依顺序用大写字母A,B,C,……编序号,如附录A,附录B,附录C,……。
    只有一个附录时也要编序号,即附录A。
    每个附录应有标题,如:“附录A 参考文献著录规则及注意事项”。

    附录一般与论文全文装订在一起,与正文一起编页码。
  
  % 致谢
  \chapterAck
    学位论文正文和附录之后,一般应放置致谢(后记或说明),主要感谢指导老师和对论文工作有直接贡献和帮助的人士和单位。致谢言语应谦虚诚恳,实事求是。字数一般不超过1000个汉字。

    “致谢”用三号黑体加粗居中,两字之间空4个半角空格。致谢内容为小四号宋体,1.5倍行距。
  

\end{document}
