\documentclass[fontset = mac ms]{seuthesis2024b}
\usepackage{lipsum}

\title{东南大学本科毕设 \LaTeX{} 模版}
% \title{东南大学本科毕设 \LaTeX{} 模版}[第二行内容]
\studentID{12345678}
\author{泰迪熊}
\department{泰迪熊学院}
\major{毛茸茸工程}
\supervisor{Ted Roosevelt}
\date{2024年1月---2024年5月}

\begin{document}
  \maketitle

  泰迪熊 Apparently, I have not finished the setup of the class ...
  
  \textsf{sans serif}

  表~\ref{tab:font-effect}。

  \begin{table}[htbp]
    \centering
    \caption{字体族、字体形状和字体系列的组合效果}
    \label{tab:font-effect}
    \begin{tabular}{c|cccc}
      & & \verb|\itshape| & \verb|\bfseries| & \verb|\itshape\bfseries| \\
      \hline
      \verb|\rmfamily| & \rmfamily 罗马体 roman & \rmfamily\itshape 倾斜 italic & \rmfamily\bfseries 加粗 bold & \rmfamily\itshape\bfseries 粗斜 bold-italic \\
      \verb|\sffamily| & \sffamily 无衬线 sans  & \sffamily\itshape 倾斜 italic & \sffamily\bfseries 加粗 bold & \sffamily\itshape\bfseries 粗斜 bold-italic \\
      \verb|\ttfamily| & \ttfamily 打字机 mono  & \ttfamily\itshape 倾斜 italic & \ttfamily\bfseries 加粗 bold & \ttfamily\itshape\bfseries 粗斜 bold-italic
    \end{tabular}
  \end{table}

  \clearpage

  \noindent
  数字运算符、化学元素符号及分子式、单位符号、产品代号、人名地名的外文字母等用正体。

  ~

  \clearpage

  摘要内容独立于正文而存在,是论文内容高度概括的简要陈述,应准确、具体、完整地概括论文的主要信息,
  内容包括研究目的、方法、过程、成果、结论及主要创新之处等,不含图表,不加注释,具有独立性和完整性,一般为400字左右。

  “摘要”用三号黑体加粗居中,“摘”与“要”之间空4个半角空格。摘要正文内容用小四号宋体,固定1.5倍行距。
  
  论文的关键词是反映毕业设计(论文)主题内容的名词,一般为3-5个,排在摘要正文部分下方。关键词与摘要之间空一行。
  关键词之间用逗号分开,最后一个关键词后不加标点符号。

  \chapter{绪论}
    这是一段文字
    这是一段文字
    这是一段文字
    这是一段文字
    这是一段文字
    这是一段文字
    这是一段文字。


    % 自动空格

    % \noindent
    % \hspace{2\ccwd}手动空格

    % {\zihao{3}\heiti\bfseries 黑体加粗 English}
    % {\heiti\bfseries 黑体加粗 English}

    \section{课题背景和意义}
      绪论部分主要论述选题的意义、国内外研究现状以及本文主要研究的内容、研究思路以及内容安排等。

      章标题为三号黑体加粗居中;一级节标题(如,2.1 本文研究内容):四号黑体居左;二级节标题(如,2.1.1 实验方法):小四号宋体居左。

      正文部分为小四号宋体,行间距1.5倍行距,首行缩进2个字符。正文一般不少于15000字。

    \section{研究现状}

    \section{本文研究内容}

      Cool \LaTeX{}!

      \lipsum

    \chapter{正文}

      具体研究内容每一章应另起页书写书写,层次要清楚,内容要有逻辑性,每一章标题需要按论文实际研究内容进行填写,不可直接写成第二章 正文。研究内容因学科、选题特点可有差异,但必须言之成理,论据可靠,严格遵循本学科国际通行的学术规范。

      中文为小四号宋体,英文及数字为小四号Times New Roman,首行缩进2个字符,行间距为1.5倍。

    \section{插图格式要求}

      插图力求精炼,且每个插图均应有图序和图名。图序与图名位于插图下方,图序一般按章节编排,如图1-1(第一章第1个图),在插图较少时可以全文连续编序,如图10。
      
      如一个插图由两个及以上的分图组成,分图用(a)、(b)、(c)等标出,并标出分图名。

      简单文字图可用 \LaTeX{} 自带宏包 Ti\textit{k}Z 直接绘制,
      复杂的图考虑使用 standalone 的 Ti\textit{k}Z 完成,以提高图形表达质量。

      插图居中排列,与上文文本之间空一行。图序图名设置为五号宋体居中,图序与图名之间空一格。
      可爱的图~\ref{fig:duck}。

      \begin{figure}[htbp]
        \centering
        \includegraphics{example-image-duck}
        \caption{经典 Ti\textit{k}Z 鸭子}
        \label{fig:duck}
      \end{figure}

      一些文字,图~\ref{fig:subfigs} 包含两个子图:
      图~\subref*{subfig:s1} 和 \subref*{subfig:s2}。

      \begin{figure}[htbp]
        \subfloat[子图一\label{subfig:s1}]{\includegraphics[width=.48\linewidth]{example-image-a}}\hfill
        \subfloat[子图二\label{subfig:s2}]{\includegraphics[width=.48\linewidth]{example-image-b}}
        \caption{两个子图}
        \label{fig:subfigs}
      \end{figure}

      \subsection{三级标题}
        三级标题内容
        \subsubsection{四级标题}
        四级标题内容
        \paragraph{段落。}内容。

      
    
    


\end{document}
