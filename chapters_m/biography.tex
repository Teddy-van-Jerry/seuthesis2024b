\begin{enumerate}
  \item 作者简介主要包括本人简历、所从事的主要研究方向和取得的科研成果,要求语句精炼,硕士生字数应控制在1000字以内,博士生字数控制在2000字以内。
  \item 作者简介内容按照下列次序编排:
  \begin{enumerate}
    \item 基本情况,包括姓名、性别、年龄、籍贯或出生地等。
    \item 从大学开始的学习和工作简历(包括毕业学校、院系、专业、学习时间;工作的单位、职务和时段等)。
    \item 本阶段学位攻读期间课程学习情况,包括学习课程的门数、总学分数、学位课程学分数等(也可以介绍各科总平均成绩情况等)。
    \item 参加研究课题(或工程设计)情况,包括课题名称、课题类别(国家级、省部级、横向协作、子课题属哪一级课题等)、研究时段、本人承担任务及完成情况。
    \item 研究生在学期间公开发表的学术论文目录和取得的其他学术成果清单(此部分单列,作为学位授予的参考条件)。
    学术论文包括期刊(杂志)论文和学术会议论文,
    其他学术成果包括学术专著、科技获奖成果、鉴定成果、专利、已完成的重要工程设计(或工程应用)项目等。
    该学术论文和成果清单所列内容需要经过导师签字认可。\newline
    按照参考文献中论文或著作的著录格式依据时间顺序排列学术著作和成果,并按顺序列出全部作者名字。
    \item 学术报告和非公开发表论文情况,以及研究生非在学期间发表的学术论文目录和取得的其他学术成果等。
  \end{enumerate}
\end{enumerate}

