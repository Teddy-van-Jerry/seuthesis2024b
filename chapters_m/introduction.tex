研究生学位论文是研究生在导师指导下从事科学研究或技术创新结果的报告和总结。

\section{学位论文的一般格式和顺序}

学位论文应包括:
(1)中文封面;
(2)中文页面;
(3)英文页面;
(4)论文独创性声明和使用授权声明;
(5)中文内容提要及关键词;
(6)英文内容提要及关键词;
(7)目录;
(8)正文;
(9)致谢;
(10)参考文献
等要素并按此顺序排列。
其他可以选择添加的内容有:位于目录之后的内容:
(11)符号、变量、缩略词等本论文专用术语的注释表;
参考文献后按序排列的内容:
(12)附录;
(13)索引(中、英文);
(14)作者简介(包括在学期间发表的论文和取得的学术成果清单);
(15)后记。

紧接英文页面之后的学位论文独创性声明和使用授权声明(见附件)需要由研究生本人亲笔签名,
学位论文需要提交电子版以便于数据库管理和网上查阅。
有保密要求不宜公开的论文由研究生申请、导师同意、院系审核,
经校保密办公室和研究生院学位办审查批准后同意保密,保密期后自动承认使用授权声明,并予以公开。

硕士学位论文一般在三万字以上,博士学位论文一般在五万字以上。文字采用中文简体;
除艺术、古籍等个别经研究生院特许的情况外,不得采用繁体字。
鼓励采用中英文双语写作,但上交国家及校图书馆的论文原则上必须用中文。
用全英文撰写的硕士学位论文需提供不少于3000字的中文摘要,
用全英文撰写的博士学位论文需提供不少于6000字的中文摘要。

对于专业学位硕士论文,可参照各院(系、所)的相关要求执行。
特殊情况可经研究生院批准后执行。

\section{绪论(前言)}
本研究课题国内外已有的重要文献的扼要概括,阐明研究此课题的目的、意义,
研究的主要内容和所要解决的问题。本研究工作在国民经济建设和社会发展中的理论意义与实用价值。

\section{文献综述}
在查阅国内外文献和了解国内外有关科技情况的基础上,围绕课题涉及的问题,
综述前人工作情况,达到承前启后的目的。要求:
(1)总结课题方向至少10年以来的国内外动态;
(2)明确前人的工作水平;
(3)介绍目前尚存在的问题;
(4)说明本课题的主攻方向。
文献总结应达到可独立成为一篇综述文章的要求。

